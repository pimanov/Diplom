\section*{Численный метод}

Численный метод, использованный в работе, описан в \cite{method}. Это конечно-разностный метод, позволяющий найти решение со вторым порядком точности по пространству и с третьим порядком точности по времени. В расчетной области вводится разнесенная, неравномерная сетка. Метод применим только в том случае, если в области можно ввести ортогональную систему координат в которой область представляет из себя паралеллепипед со стенками, параллельными координатным осям. То есть, если $\Omega$~--- расчетная область, то существуют такие отрезки $l_1, l_2, l_3$, что $\Omega = l_1 \times l_2 \times l_3$. 

Под разностной схемой, построенной на разнесенных сетках\footnote{так же разнесенные сетк называют перемежающиеся сетки, или смещенные сетки, Angl.: staggered mash.}, понимают такую, в которой разные неизвестные определины в разных узлах. В нашем случае необходимо определить три компоненты ветора скорости, три компоненты ветора завихренности и двление. 
, на прямойгольной сетке, давление определено в центрах ячеек, компонениы вектора скороти определены в центрах граней ячейки, а компоненты вектора завихренности~--- в центрах ребер. Если выразить в проекции на каждую ось координат, то давление всегда определяется в центре ячейки, на i-ой координатной оси  компонента вектора скорости определяется в узлах на i-ой оси и в центрах ячеек на двух других осях, i-ая компонента вектора завихренности определяется в центрах ячеек на i-ой иси 

\begin{figure}
  \begin{center}
    \begin{picture}(180,180)(-90,-90)
      \thinlines
     \put(-90,10){\vector(0,1){40}}
     \put(-90,10){\vector(1,0){40}}
     \put(-90,10){\vector(-1,-1){20}}
     \put(-97,46){\text{z}}
     \put(-56,3){\text{y}}
     \put(-110,-2){\text{x}}
      \thicklines
     \put(-30,-30){\line(0,1){60}}
     \put(-30,30){\line(1,0){60}}
     \put(30,30){\line(0,-1){60}}
     \put(30,-30){\line(-1,0){60}}
     %
     \put(-30,30){\line(1,1){24}}
     \put(30,30){\line(1,1){24}}
     \put(30,-30){\line(1,1){24}}
     %
     \put(-6,54){\line(1,0){60}}
     \put(54,54){\line(0,-1){60}}
      \thinlines
%     \put(-30,-30){\line(1,1){24}}
%    \put(-6,-6){\line(1,0){60}}
%     \put(-6,-6){\line(0,1){60}}
      \thicklines
     \put(0,0){\circle{6}}
     \put(3,-7){\text{$v_x$}}
     \put(42,12){\circle{6}}
     \put(44,5){\text{$v_y$}}
     \put(12,42){\circle{6}}
     \put(15,35){\text{$v_z$}}
     \put(-30,0){\circle*{4}}
     \put(-27,-7){\text{$\omega_z$}}
     \put(0,30){\circle*{4}}
     \put(3,23){\text{$\omega_y$}}
     \put(0,-30){\circle*{4}}
     \put(3,-37){\text{$\omega_y$}}
     \put(30,0){\circle*{4}}
     \put(33,-7){\text{$\omega_z$}}
     \put(54,24){\circle*{4}}
     \put(57,17){\text{$\omega_z$}}
     \put(42,-18){\circle*{4}}
     \put(45,-25){\text{$\omega_x$}}
     \put(42,42){\circle*{4}}
     \put(43,35){\text{$\omega_x$}}
     \put(24,54){\circle*{4}}
     \put(27,47){\text{$\omega_y$}}
     \put(-18,42){\circle*{4}}
     \put(-15,35){\text{$\omega_x$}}
    \end{picture}
  \end{center}
  \caption{Смещенные сетки. Изображена одная ячейка сетки. Давление p определяется в центре ячейки.}
  \label{picStag}
\end{figure}


Для того, что бы ввести неравномерную сетку, использу монотонную функцию преоюразования $ x = x(\xi) $  , отображающею отрезок [0,1] в отрезок [0,1]. 
\begin{gather}
  x(\xi): [0,1] \longrightarrow [0,1]
\end{gather}
Такая функция переведет равномерную стетку, введеную на отрезке [0,1] в неравномерную. 


Для того, чтобы выписать разностную аппроксимацию исходной системы уравнений, достаточно определить разностную производную первого порядка и осреднение по прострвнству. Для произвольной функции $f(x)$ опеределен разностный оператор дифференцирования $ $ 