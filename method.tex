\section*{Численный метод}

Численный метод, использованный в работе, описан в \cite{method}. Это конечно-разностный метод, позволяющий найти решение со вторым порядком точности по пространству и с третьим порядком точности по времени. В расчетной области вводится разнесенная, неравномерная сетка. Метод применим только в том случае, если в области можно ввести ортогональную систему координат в которой область представляет из себя паралеллепипед со стенками, параллельными координатным осям. То есть, если $\Omega$~--- расчетная область, то существуют такие отрезки $l_1, l_2, l_3$, что $\Omega = l_1 \times l_2 \times l_3$. 

Под разностной схемой, построенным на разнесенных сетках\footnote{так же разнесенные сетк называют перемежающиеся сетки, или смещенные сетки, Angl.: staggered mash.} понимают такую, в которой разные физические величины определины в разных узлах. В нашем случае

Для того, что бы ввести неравномерную сетку, использу монотонную функцию преоюразования $ x = x(\xi) $  , отображающею отрезок [0,1] в отрезок [0,1]. 
\begin{gather}
  x(\xi): [0,1] \longrightarrow [0,1]
\end{gather}
Такая функция переведет равномерную стетку, введеную на отрезке [0,1] в неравномерную