\section*{Введение \cite{introduction}}

Исследование течения в каверне с подвижной крышкой~--- это классическая задача вычислительной гидродинамики. Это одна из тех задачь, на которою обратили свое внимание первые вычислители, и до сих пор к ней не угасает интерес. Этому есть три причины. Во-первых, подобные задачи возникают во многих производственных процесах, например, в лакокрасочной отрасли. Во-вторых, в данной задаче возникают и могут быть изучены такие процессы возникновения и эволюции визря или пограничного слоя, при том, что все линии тока замкнуты. Наконец, в правильной геометрии легко реализовать различные численные методы для их верификации и отладки.

Считается, что вычислительная гидродинамика началась с исследованей Кима и Моина \cite{KimMoin}

 
\newpage
