\subsection{Математическая постановка задачи}

Ставиться двумерная задача рачета поля скорости жидкости $ \vec v(t) $ в каверне $ \Omega $ 
с подвижной крышкой. Каверна имеет форму квадрата $ \Omega = (0,1) \times (0,1) $. 
Жидкости передается движение крышки $ \Gamma_1 = (0,1) \times {1} $. 
Крышка движется равномерно со скорость $ \vec{v}_0 = (1,0) $. 
Считается, что жидкость вязкая, ньютоновская, несжимаемая, изотермическая, и описывается системой
уравнений Навье-Стокса.
\begin{gather}
  \nabla \cdot \vec v = 0 \\
  \vec \omega = \nabla \times \vec v \\
  \frac{\partial \vec v}{\partial t} = \vec v \times \vec \omega - \nabla p - 
  \nu ( \nabla \times \vec \omega ), \text{при } \vec x \in \Omega \\
  \vec v = (1,0), \text{при } \vec x \in \Gamma_1 \\
  \vec v = (0,0), \text{при } \vec x \in \Gamma_2 \\
  \vec v (0) = \vec v _0
\end{gather}

Здесь $ \vec \omega $ - завихренность, p - полное кинематическое давление,
$ \nu = 1 / \Re $ - вязкость, $ \Gamma_2 = \partial \Omega \setminus \Gamma_1 $ - 
боковые стенки и дно каверны, $ \partial \Omega = \Gamma = \Gamma_1 \cup \Gamma_2 $ - 
граница области, $\vec v _0$ - начальное распредиление скорости. 

\begin{center}
\begin{picture}(150,110)(-75,-55)
\linethickness{0.4pt}
\put(-80,-40){\vector(1,0){160}}
\put(-40,-80){\vector(0,1){160}}
\put(-49,-51){\textsl{0}}
\put(37,-51){\textsl{1}}
\put(-49,30){\textsl{1}}
\put(80,-51){\textsl{x}}
\put(-49,80){\textsl{y}}
\linethickness{1.0pt}
\put(-40,-40){\line(1,0){80}}
\put(40,-40){\line(0,1){80}}
\put(60,40){\line(-1,0){120}}
\put(60,60){\line(-1,0){120}}
\put(-40,40){\line(0,-1){80}}
\put(-60,50){\circle{20}}
\put(-60,50){\oval(140,20)}
\end{picture}
\end{center}