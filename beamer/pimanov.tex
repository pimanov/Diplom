\documentclass{beamer}
\usepackage[english,russian]{babel}
\usepackage[utf8]{inputenc}
\usepackage{tikz}
\usepackage{graphicx}
\usepackage{epstopdf}
 \usepackage{pgfplots}
 \renewcommand{\Re}{\mathop{\mathrm{Re}}\nolimits}

% Стиль презентации
\usetheme{Copenhagen}
\begin{document}
\institute{МОСКОВСКИЙ ГОСУДАРСТВЕННЫЙ УНИВЕРСИТЕТ имени.~М.{\,}В.{\,}Ломоносова\\
	\vspace{0.7cm} \smallНаучные руководители:\\[-7mm]
	\vspace{0.7cm} к.ф.-м.н. И.{\,}С.{\,}Калачинская,\\[-7mm]
	\vspace{0.7cm} кафедра ВМ, ВМК МГУ;\\[-7mm]
	\vspace{0.7cm} д.ф.-м.н. Н.{\,}В.{\,}Никитин,\\[-7mm]
	\vspace{0.7cm} НИИ Механики МГУ\\
}

\title[]{Численное исследование 3-х мерных не изотермических течений несжимаемой жидкости в прямоугольной каверне с подвижной крышкой}
 
\author{Пиманов Владимир, группа 505}


\date{Москва 2013} 
% Создание заглавной страницы
\frame{\titlepage} 
% Автоматическая генерация содержания

\begin{frame}{Физическая постановка задачи}
	\begin{block}{Жидкость}
		вязкая, ньютоновская, несжимаемая - описывается системой уравнений Навье-Стокса.
	\end{block}
\begin{block}{Каверна}
		\tikz[scale=1.2]{
			\draw [fill=blue!20](0,0)--(1,0)--(1,1)--(0.9,1)-- (0.9,0.1)--(0.1,0.1)--(0.1,1)--(0,1)--(0,0) ;
			\draw [fill=blue!20](0,1.3)--(2,1.3)--(2,1.4)--(0,1.4)-- (0,1.3) ;
			\draw (2,1.3)--(3,2.3) ;
			\draw (1,0)--(3,2) ;
			\draw (2,1.4)--(3,2.4) ;
			\draw (0,1.4)--(1,2.4) ;
			\draw (1,1)--(1.3,1.3) ;
			\draw (0.9,1)--(1.2,1.3) ;
			\draw (0.1,1)--(0.4,1.3) ;
			\draw (0,1)--(0.3,1.3) ;
			\draw (0.1,0.1)--(0.9,0.9) ;
			\draw [->] (-1,-1) -- (-1,0);
			\draw [->] (-1,-1) -- (0,-1);
			\draw [->] (-1,-1) -- (-0.3,-0.3);
			\draw [->] (2.5,1.85) -- (3.5,1.85);
			\draw (-1.2,-0.5) node [above]  {x} ;
			\draw (0,-1) node [above]  {y} ;
			\draw (-0.5,-0.3) node [above]  {z} ;
			\draw (-4,2.5) node [above]  { } ;
		}
	\end{block}
\end{frame}

\begin{frame}{Цели работы}
	
		Всегда существует двумерное течение, возможно неустойчивое. Такое двумерное течение будем называть \textit{основным течением}. Основные течения уже хорошо изучены, как численно, так и эксперементально. Цель данной работы: 
	\begin{itemize}
		\item расчитать основные течения при некоторых числах Рейнольдса,
		\item исследовать основное течение на устойчивость к малым трехмерным возмущениям,
		\item определить пороговое число Рейнольдса, при котором основное течение теряет устойчивость.
	\end{itemize}
\end{frame}

\begin{frame}{Математическая постановка задачи}
	\begin{block}{Система уравнений Навье--Стокса}
		\begin{gather*}
			\nabla \cdot \vec v = 0 \\
			\frac{\partial \vec v}{\partial t} = \vec v \times \vec \omega - \nabla p - 
			\nu ( \nabla \times \vec \omega ) 
		\end{gather*}
		$$
			\text{Здесь } \vec v \text{ --- искомое поле скорости,} 
		$$ $$
			\vec \omega = \nabla \times \vec v \text{ --- завихренность поля скорости.}
		$$
	\end{block}

	\begin{block}{Граничные и начальные условия}
		\begin{gather*}
			\vec v = (1,0), \text{ на верхней крышке} \\
			\vec v = (0,0), \text{ на боковых стенках и дне}\\
			\vec v (0) = \vec v _0
		\end{gather*}
	\end{block}

\end{frame}

\begin{frame}{Система уравнений для поиска основного течения}
	\begin{block}{}
		Полагая, что от переменной z ничего не зависит и проекция скорости $\vec v$ на ось Oz равна нулю, преобразуем исходную систему 
		уравнений в систему для поиска основного течения. 
		\begin{gather*}
			\nabla \cdot \vec v = 0 \\
			0 = \vec v \times \vec \omega - \nabla p - \nu ( \nabla \times \vec \omega ) \\
		\end{gather*}
	\end{block}

	\begin{block}{Граничные условия}
		\begin{gather*}
			\vec v = (1,0), \text{ на верхней крышке} \\
			\vec v = (0,0), \text{ на боковых стенках и дне}
		\end{gather*}
	\end{block}
\end{frame}

\begin{frame}{Линеаризованная система уравнений для расчета эволюции малых трехмерных возмущений}
	\begin{block}{}
		\begin{gather*} 
 		 	\nabla_\alpha^* \cdot \vec v = 0 \\
   			\frac{\partial \vec v}{\partial t} = \vec V \times \vec \omega + \vec v \times \vec \Omega - 
			\nabla_\alpha p - \nu ( \nabla \times \vec \omega ) \\
			\text{	Здесь }   \nabla_\alpha = (\frac{\partial}{\partial x},\frac{\partial}{\partial y},-\alpha I), 
  \nabla_\alpha^* = (\frac{\partial}{\partial x},\frac{\partial}{\partial y},\alpha I) 
		\end{gather*}
	\end{block}

	\begin{block}{Граничные условия}
		\begin{gather*}
	  		\vec v = \vec 0, \text{ на границе} \\
  			\vec v (0) = \vec v _0 
		\end{gather*}
	\end{block}
\end{frame}

\begin{frame}{Численый метод решения двумерной задачи}
	\begin{block}{}
		При решении системы уравнений Навье--Стокса применялся конечно--разностный метод со вторым порядком точности по шагу сетки. 
		Использовались  разнесенные, неравномерные сетки. Для интегрирования по времени применялся полу неявный метод Рунге--Кутты 
		третьего порядка.  
	\end{block}
	\begin{block}{Разнесенные сетки}
		Разные физические виличины относятся к разным узлам сетки. 
	\end{block}

	\begin{block}{Неравномерная сетка}
		Основные гидродинамические явления происходят вдоль стенки и есть смысл уплотнить сетку около сетнок, в то время, как в большей части каверны сетка останется разреженной.
	\end{block}
\end{frame}

\begin{frame}{Численное решение линеаризованной системы уравнений}
	\begin{block}	{}
		При интегрировании по времени некоторых начальных возмущений, наблюдается их асимптотический экспоненциальный рост либо затухание.
		При тех числах Рейнольдса, при которых хотя бы при одном волновом числе наблюдается асимптотический рост возмущений, 
		решение неустойчиво к малым трехмерным возмущениям. Если при некотором числе Рейнольдса при каждом волновом числе наблюдается 
		асимптотическое затухание, то при этом числе Рейнольдса решение устойчиво. 

		Критическое число Рейнольдса~--- то число Рейнольдса, при котором основное течение теряет устойчивость.
	\end{block}
\end{frame}

\begin{frame}{Решение двумерной задачи}
	\begin{block}{Сравнение результатов с эталонным решением}
		\begin{table}[htp]
			\center
  				\begin{tabular}{lcccc}
			\hline
				Re = 1000 
				& $\psi$   	& $\omega$ 	& x 		& y \\
			\hline	
				Benchmark 	& 0.118942 	& 2.067213 	& 0.5300   	& 0.5650 \\
				Present		& 0.11523	& 2.06984	& 0.539		& 0.563 \\
			\hline 
			\hline
				Re = 5000 
				& $\psi$   	& $\omega$ 	& x 		& y \\
			\hline	
				Benchmark 	& 0.122233 	& 1.940732 	& 0.5150   	& 0.5433 \\
				Present		& 0.1236	& 1.9387	& 0.514		& 0.550 \\
			\hline 
  	 \end{tabular}
		\caption{Значения максимума функции тока $\psi$, значение завихренности $\omega$ в той же точке и координаты этой точки для 
		представленного метода (Present) и эталонного решения (Benchmark). При Re = 1000 и Re = 5000. Крышка движется слева направо.}
		\end{table}
	\end{block}
\end{frame}

\begin{frame}{Решение двумерной задачи}
	\begin{block}{}
		\begin{figure}
		\centering
		\includegraphics[width = 0.9\linewidth]{1000.pdf}
	
		\includegraphics[width = 0.9\linewidth]{5000.pdf}
		\caption{Изображены слева направо функция тока, завихренность и поле давления. Re=1000, Re=5000}
		\end{figure}
	\end{block}
\end{frame}


\begin{frame}{Результаты}
	\begin{figure}
  \center
  \begin{tikzpicture}
    \begin{axis}[
% 	xmode = log, ymode = log,
%	xmin = 0, ymin = 0,
	xlabel=$\alpha$,
	ylabel=$\Re_{crit}$ ]
      \addplot coordinates {
%	(3, 4800) (4, 4800) (5, 1950)
   (6, 1050) (7, 950) (8, 950) (9, 1050) (10, 1150)
	(11, 1050) (12, 950) (13, 850) (14, 800) (15, 790) (16, 790) (17, 800) (18, 850)
	(19, 850) (20, 900)};
    \end{axis}
  \end{tikzpicture}
  
  \caption{Кривая нейтральной устойчивости}
  \label{graph:Re_al}
\end{figure}
\end{frame}


\begin{frame}{Заключение}
	\begin{block}{}
		\begin{itemize}
			\item Разработан программный комплекс для нахождения основного течения и интегрироавния по времени малых трехмерных возмущений
 			\item Построена кривая нейтральной устойчивости и найдено критическое число Рейнольдса.
			\item Было продемонстрироавно хорошее соответствие полученных результатов результатам других авторов.
		\end{itemize}
	\end{block} 
\end{frame}

\end{document}